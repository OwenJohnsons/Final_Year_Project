\section{Introduction \& Background} \label{sc: Introduction}

Clusters of stars are one of the most readily available ‘laboratories’ to give insight into many Astrophysical phenomena. Study and analysis of clusters lend themselves to the development of theories in the stellar and galactic evolution, along with an insight into stellar composition and nuclear synthesis but have uses across many fields in astrophysics. \\
Open clusters are classified based on their sparseness with some distinction about their core, with a higher density usually observed towards their centre, with subcategories within classified open clusters. \cite{nilakshi_sagar_pandey_mohan_2002, trumpler_1930}

\subsection{Classification of Open Clusters}

When performing analysis it's important to know the classification of a open cluster. The most common system for doing this was coined by Robert Trumpler who determined that a open cluster could be classified based on three factors. \textbf{\textit{(a)}} Range of brightness, \textbf{\textit{(b)}} degree of concentration and \textbf{\textit{(c)}} star population in a cluster\footnote{'Nebulosity' is a term used to describe when a cluster has similar to a nebula such as a cloud like properties.}.

\begin{table}[h!]
    \centering
    \begin{tabularx}{\textwidth}{XXX}
        \textbf{Range of Brightness} & \textbf{Degree of Concentration} & \textbf{Cluster population} \\ \hline 
        1 - Majority of stellar objects show similar brightness. & I - Strong central concentration (Detached) & p - Poor $(n < 50)$  \\
        2 - Moderate brightness ranges between stellar objects. & II - Little central concentration (Detached)& m - Medium $(50 < n < 100)$ \\
        3 - Both bright and feint stellar objects  & III - No disenable concentration & r - Rich $(n > 100)$\\ 
          & IV - Clusters not well detached (Strong field concentration) & \\ \hline \hline 
    \end{tabularx}
    \caption{Details relating to the classification of open clusters as described by the Trumpson classification system \cite{trumpler_1930}. Where $n$ denotes the amounts the stellar population in a given cluster. For example Pleiades is a I3rn cluster and Hyades is a II3m cluster. Where the 'n' flag on a classification relates if the cluster shows nebulosity.}
\end{table}

\subsection{Existing Catalogues \& Survey's}

There have been many initiatives to catalogue open clusters throughout the years. In the case of cataloguing open clusters the conditions for membership must be well defined and understood as they directly change ideas about stellar and galactic evolution. As time progresses the realms in which these conditions are defined become less transparent. 




\section{Using Open Clusters as Stellar Laboratories}
Cluster's form the perfect environment for large scale laboratories. This initially became prevalent when examining the color magnitude diagrams of open clusters. Stars in the same cluster were found to often have similar properties across the populations \cite{trumpler_1930} allowing for details of the molecular cloud to hold true for detailed observations. \\ So when a H-R diagram is plotted for an open cluster the stellar population will mainly reside along the main sequence. 


\section{Use of Open Clusters in Galactic evolution }