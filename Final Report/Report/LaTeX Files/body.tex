\section*{}

% \begin{auxmulticols}{1}
%     \lipsum[1-2]
% \end{auxmulticols}

\begin{center}
    \textit{This work is dedicated to my mentor and my friend Noel White, (1950-2021).}
\end{center}

\section{Introduction} \label{sec:intro}

Open clusters have been shown to be an integral part of the astronomers toolbox, readily lending themselves as stellar laboratories. Open clusters are classified as a group of stars around the same age and loosely bound through mutual gravitation. 
Their similar age allows for in depth observation of the stellar evolution.  Through this many attributes of the stellar population can be inferred. As clusters span age ranges from a X to X, many have been present since formation of the disk itself. Through this if clusters of varying ages are examined it's possible to trace out the evolution of the milky way.  \\

Mapping the milky way has always been difficult given the vantage point it can be observed from. This makes it quite difficult to appreciate the shape and dimensions of the milky way. Some of the pioneering studies such as \cite{1785RSPT...75..213H,1918ApJ....48..154S} and \cite{1930LicOB..14..154T} first outline the use of open clusters to map the galaxy. Following with studies like \cite{1970IAUS...38..205B} which pathed the spiral arms of the milky way using open clusters and numerous studies by \cite{1958ZA.....46..176V} which explore the evolution of the galaxies scale height. To more recent studies by X \\

While the precision and accuracy of cluster age estimates are tied to the quality of the observational data and theoretical models the process of estimating cluster age through use of colour-magnitude diagrams is relativity straightforward  and been shown to be tried and true. Even early open cluster catalogues like X and X included distance estimates while more recent catalogues like X and X have provided other parameters such as age, metallicity  and excess colour. Furthermore with the second data release from GAIA (X) presents the most in-depth all sky astrometric and photometric study to date. \\
This increase in available data has allowed for the characterisation of open clusters on mass adding to catalogues such as WEBDA. Determination of all open clusters identified by Gaia is an ongoing task and is being automated using modern techniques and machine learning as shown in studies by X and X. \\

This study used the 1.25 m optical telescope at the Calar Alto Observatory (CAHA) to observe four open clusters from the WEBDA catalogue. The aim of this work was to classify the four observed clusters and infer details of each cluster. Then use this observational cluster classification in tandem with other open clusters from the WEBDA catalogue to trace the paths of clusters in the galactic disk studying both its structure and evolution.

\section{Observations} 

\subsection{Photometry}

\subsection{Astrometry}

\section{Supplementary Data}