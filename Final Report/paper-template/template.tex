\documentclass[twocolumn]{aastex63}

\newcommand{\vdag}{(v)^\dagger}
\newcommand\aastex{AAS\TeX}
\newcommand\latex{La\TeX}

\shorttitle{Short title}
\shortauthors{O. Johnson}

\graphicspath{{./}{figures/}}


\begin{document}

\title{Project title}


\author[0000-0002-5927-0481]{Owen Johnson}
\affiliation{School of Physics, University College Dublin, Ireland}


\begin{abstract}

Abstract must contain the context of your project, aims, info about the methods, results and conclusions. It should be limited to 300 words.
Do no change the format of the paper (margins, font size, two column for main text, one column for Appendix...)

\end{abstract}

\keywords{type of source, phenomena studied ... }

\section{Introduction} \label{sec:intro}

If you want to do a footnote with a url, please do it like this\footnote{\url{http://www.latex-project.org/}}.

For references cited that belong within brackets, use the command citep for a result like this: \citep{lamport94}. 

If the reference is part of the text, then use cite, and the reference would look like this: \cite{1989BAAS...21..780H}.

If you need to cite two references together, the best way to do it is like this: \citep{{lamport94},{1989BAAS...21..780H}}. This is normally done, when two or more papers referred to the statement of example that we were talking about on the text. You have probably seen it on your literature review.

All references must be saved on a bib file (see example). NASA/ADS provides with the all information needed to be added to the bib file. All you have to do is find the reference you want to add on NASA/ADS, look for bib file information and copy the output given by NASA/ADS into your bib file.

Most of your references would belong to the introduction. Here, you need to set the environment for your project. Talk about what is relevant to your project and end the section by describing in a short paragraph how the paper is structured.

Typical structure of a paper goes like this: Introduction, Observations and data analysis, results, discussion, and finally conclusions. If there is something that requires a very detailed explanation and breaks the flow of the paper, then that information should be provided as an appendix.

You can add more sections or subsections depending on your needs. For example, if you are doing some physical modelling, you may want to leave the results to the more direct output of your analysis and add a new section where you do the modelling. Or you can have that as a subsection. That is up to you.

\section{Manuscript styles} \label{sec:style}

Note that in the two column style figures and tables will only
span one column unless specifically ordered across both with the ``*'' flag,
e.g. \\

\noindent{\tt\string\begin\{figure*\}} ... {\tt\string\end\{figure*\}}, \\
\noindent{\tt\string\begin\{table*\}} ... {\tt\string\end\{table*\}}. \\

\section{Floats} \label{sec:floats}

Floats are non-text items that generally can not be split over a page.
They also have captions and can be numbered for reference.  Primarily these
are figures and tables but authors can define their own. 

If you want to take the time to optimize the locations of
their floats there are some techniques that can be used.  The simplest
solution is to placing a float earlier in the text to get the position
right but this option will break down if the manuscript is altered.
A better method is to force \latex\ to place a
float in a general area with the use of the optional {\tt\string [placement
specifier]} parameter for figures and tables. This parameter goes after
{\tt\string \begin\{figure\}}, {\tt\string \begin\{table\}}, and
{\tt\string \begin\{deluxetable\}}.  The main arguments the specifier takes
are ``h'', ``t'', ``b'', and ``!''.  These tell \latex\ to place the float
\underline{h}ere (or as close as possible to this location as possible), at
the \underline{t}op of the page, and at the \underline{b}ottom of the page.
The last argument, ``!'', tells \latex\ to override its internal method of
calculating the float position.  A sequence of rules can be created by
using multiple arguments.  For example, {\tt\string \begin\{figure\}[htb!]}
tells \latex\ to try the current location first, then the top of the page
and finally the bottom of the page without regard to what it thinks the
proper position should be.  Many of the tables and figures in this article
use a placement specifier to set their positions.

Note that the \latex\ {\tt\string tabular} environment is not a float.  Only
when a {\tt\string tabular} is surrounded by {\tt\string\begin\{table\}} ...
{\tt\string\end\{table\}} is it a true float and the rules and suggestions
above apply.

You might have to use {\tt\string\clearpage} to
isolate a long table or optimally place it within the surrounding text.

\subsection{Tables} \label{subsec:tables}

Table \ref{tab:messier} can be used as an indication on the table layout. Please follow these rules to make all the tables look similar/consistent among papers. 

It also shows the option of using a deluxetable, which comes with some additional features over normal tables. Please do use whatever type of table you want, but as mentioned, keep the layout the same.

The Common name column is the third in the \latex\ deluxetable but does not appear when the article
is compiled. This hidden column can be shown simply by changing the ``h'' in
the column identifier preamble to another valid value.  This table also
uses {\tt\string\tablenum} to renumber the table because a \latex\ tabular
table was inserted before it.

\begin{deluxetable*}{cchlDlc}
\tablenum{1}
\tablecaption{Fun facts about the first 10 messier objects\label{tab:messier}}
\tablewidth{0pt}
\tablehead{
\colhead{Messier} & \colhead{NGC/IC} & \nocolhead{Common} & \colhead{Object} &
\multicolumn2c{Distance} & \colhead{} & \colhead{V} \\
\colhead{Number} & \colhead{Number} & \nocolhead{Name} & \colhead{Type} &
\multicolumn2c{(kpc)} & \colhead{Constellation} & \colhead{(mag)}
}
\decimalcolnumbers
\startdata
M1 & NGC 1952 & Crab Nebula & Supernova remnant & 2 & Taurus & 8.4 \\
M2 & NGC 7089 & Messier 2 & Cluster, globular & 11.5 & Aquarius & 6.3 \\
M3 & NGC 5272 & Messier 3 & Cluster, globular & 10.4 & Canes Venatici &  6.2 \\
M4 & NGC 6121 & Messier 4 & Cluster, globular & 2.2 & Scorpius & 5.9 \\
M5 & NGC 5904 & Messier 5 & Cluster, globular & 24.5 & Serpens & 5.9 \\
M6 & NGC 6405 & Butterfly Cluster & Cluster, open & 0.31 & Scorpius & 4.2 \\
M7 & NGC 6475 & Ptolemy Cluster & Cluster, open & 0.3 & Scorpius & 3.3 \\
M8 & NGC 6523 & Lagoon Nebula & Nebula with cluster & 1.25 & Sagittarius & 6.0 \\
M9 & NGC 6333 & Messier 9 & Cluster, globular & 7.91 & Ophiuchus & 8.4 \\
M10 & NGC 6254 & Messier 10 & Cluster, globular & 4.42 & Ophiuchus & 6.4 \\
\enddata
\tablecomments{This table ``hides'' the third column in the \latex\ when compiled.
The Distance is also centered on the decimals.  Note that when using decimal
alignment you need to include the {\tt\string\decimals} command before
{\tt\string\startdata} and all of the values in that column have to have a
space before the next ampersand.}
\end{deluxetable*}

\subsection{Figures\label{subsec:figures}}

\begin{figure}[ht!]
\plotone{cost.pdf}
\caption{The subscription (squares) and author publication (asterisks) 
costs from 1991 to 2013. Subscription cost are on the left Y axis while
the author costs are on the right Y axis. All numbers in US dollars and
adjusted for inflation. The author charges also account for the change
from page charges to digital quanta in April 2011.  \label{fig:general}}
\end{figure}

All figures should be added as PDFs and should
include detailed and descriptive captions.

\subsection{General figures\label{subsec:general}}

There are multiple ways of adding figures. Figure \ref{fig:general} is an example which shows an example using the special command {\tt\string\plotone}. For a general figure consisting of two PDFs files the {\tt\string\plottwo} command can be used to position the two image files side by side.

Both {\tt\string\plotone} and {\tt\string\plottwo} take a
{\tt\string\caption}.  Each is
based on the {\tt\string graphicx} package command,
{\tt\string\includegraphics}.  

You are welcome to use
{\tt\string\includegraphics} along with its optional arguments that control
the height, width, scale, and position angle of a file within the figure.

\subsection{Grid figures}

If you want to include more than two PDF files in a single figure you can use
{\tt\string\gridline} which allows any number of individual PDF file
calls within a single figure.  Each file cited in a {\tt\string\gridline}
will be displayed in a row.  By adding more {\tt\string\gridline} calls an
author can easily construct a matrix X by Y individual files as a
single general figure.

For each {\tt\string\gridline} command a PDF file is called by one of
four different commands.  These are {\tt\string\fig},
{\tt\string\rightfig}, {\tt\string\leftfig}, and {\tt\string\boxedfig}.
The first file call specifies no image position justification while the
next two will right and left justify the image, respectively.  The
{\tt\string\boxedfig} is similar to {\tt\string\fig} except that a box is
drawn around the figure file when displayed. Each of these commands takes
three arguments.  The first is the file name.  The second is the width that
file should be displayed at.  While any natural \latex\ unit is allowed, it
is recommended to use fractional units with the
{\tt\string\textwidth}.  The last argument is text for a subcaption.

Figure \ref{fig:pyramid} is a good example of this option.

\begin{figure*}
\gridline{\fig{V2491_Cyg.pdf}{0.3\textwidth}{(a)}
          \fig{HV_Cet.pdf}{0.3\textwidth}{(b)}
          \fig{LMC_2009.pdf}{0.3\textwidth}{(c)}
          }
\gridline{\fig{RS_Oph.pdf}{0.3\textwidth}{(d)}
          \fig{U_Sco.pdf}{0.3\textwidth}{(e)}
          }
\gridline{\fig{KT_Eri.pdf}{0.3\textwidth}{(f)}}
\caption{Inverted pyramid figure of six individual files. The nova are
(a) V2491 Cyg, (b) HV Cet, (c) LMC 2009, (d) RS Oph, (e) U Sco, and (f) 
KT Eri. These individual figures are taken from \citet{2011ApJS..197...31S}.
\label{fig:pyramid}}
\end{figure*}

\section{Displaying mathematics} \label{sec:displaymath}

Mathematics can be displayed either within the text, e.g. $E = mc^2$, or
separate from in an equation.  In order to be properly rendered, all inline
math text has to be declared by surrounding the math by dollar signs (\$).

A complex equation example with inline math as part of the explanation
follows.

\begin{equation}
\bar v(p_2,\sigma_2)P_{-\tau}\hat a_1\hat a_2\cdots
\hat a_nu(p_1,\sigma_1) ,
\end{equation}
where $p$ and $\sigma$ label the initial $e^{\pm}$ four-momenta
and helicities $(\sigma = \pm 1)$, $\hat a_i=a^\mu_i\gamma_\nu$
and $P_\tau=\frac{1}{2}(1+\tau\gamma_5)$ is a chirality projection
operator $(\tau = \pm1)$.  This produces a single line formula. Note the coma following the equation indicating is part of the sentence. If the equation ends the sentence, then a period would be expected.

\latex\ can also handle a a multi-line equation.  Use {\tt\string eqnarray}
for more than one line and end each line with a
\textbackslash\textbackslash.  Each line will be numbered unless the
\textbackslash\textbackslash\ is preceded by a {\tt\string\nonumber}
command.  Alignment points can be added with ampersands (\&).  There should be
two ampersands per line. In the examples they are centered on the equal
symbol.

\begin{eqnarray}
\gamma^\mu  & = &
 \left(
\begin{array}{cc}
0 & \sigma^\mu_+ \\
\sigma^\mu_- & 0
\end{array}     \right) ,
 \gamma^5= \left(
\begin{array}{cc}
-1 &   0\\
0 &   1
\end{array}     \right)  , \\
\sigma^\mu_{\pm}  & = &   ({\bf 1} ,\pm \sigma) , 
\end{eqnarray}

\begin{eqnarray}
\hat a & = & \left(
\begin{array}{cc}
0 & (\hat a)_+\\
(\hat a)_- & 0
\end{array}\right), \nonumber \\
(\hat a)_\pm & = & a_\mu\sigma^\mu_\pm 
\end{eqnarray}


\section{Software and third party data repository citations} \label{sec:cite}

Software such as Astropy, matploblib... and particular modules used for your analysis should be cited. Same goes for any catalogue used during your analysis. Some of these packages have papers that they recommend to use for citations. In other cases, an acknowledgement on the acknowledgements section could be enough.

\acknowledgments

Thank your dog, friends and family for whatever reason. You may want to thank the staff at Calar Alto for their help if you want to.
Any other acknowledgement of things not included on the bibliography that you used should be added here.


%% Appendix material should be preceded with a single \appendix command.
%% There should be a \section command for each appendix. Mark appendix
%% subsections with the same markup you use in the main body of the paper.

%% Each Appendix (indicated with \section) will be lettered A, B, C, etc.
%% The equation counter will reset when it encounters the \appendix
%% command and will number appendix equations (A1), (A2), etc. The
%% Figure and Table counter will not reset.

\appendix

\section{Appendix information}

Appendices can be broken into separate sections just like in the main text.
The only difference is that each appendix section is indexed by a letter
(A, B, C, etc.) instead of a number.  Likewise numbered equations have
the section letter appended.  Here is an equation as an example.

\begin{equation}
I = \frac{1}{1 + d_{1}^{P (1 + d_{2} )}}
\end{equation}

Appendix tables and figures should not be numbered like equations. Instead
they should continue the sequence from the main article body.

\section{Rotating tables} \label{sec:rotate}

To place a single page table in a
landscape mode start the table portion with
{\tt\string\begin\{rotatetable\}} and end with
{\tt\string\end\{rotatetable\}}.

Tables that exceed a print page take a slightly different environment since
both rotation and long table printing are required. In these cases start
with {\tt\string\begin\{longrotatetable\}} and end with
{\tt\string\end\{longrotatetable\}}. Table \ref{chartable} is an
example of a multi-page, rotated table. The {\tt\string\movetabledown}
command can be used to help center extremely wide, landscape tables. The
command {\tt\string\movetabledown=1in} will move any rotated table down 1
inch. 

\begin{longrotatetable}
\begin{deluxetable*}{lllrrrrrrll}
\tablecaption{Observable Characteristics of 
Galactic/Magellanic Cloud novae with X-ray observations\label{chartable}}
\tablewidth{700pt}
\tabletypesize{\scriptsize}
\tablehead{
\colhead{Name} & \colhead{V$_{max}$} & 
\colhead{Date} & \colhead{t$_2$} & 
\colhead{FWHM} & \colhead{E(B-V)} & 
\colhead{N$_H$} & \colhead{Period} & 
\colhead{D} & \colhead{Dust?} & \colhead{RN?} \\ 
\colhead{} & \colhead{(mag)} & \colhead{(JD)} & \colhead{(d)} & 
\colhead{(km s$^{-1}$)} & \colhead{(mag)} & \colhead{(cm$^{-2}$)} &
\colhead{(d)} & \colhead{(kpc)} & \colhead{} & \colhead{}
} 
\startdata
CI Aql & 8.83 (1) & 2451665.5 (1) & 32 (2) & 2300 (3) & 0.8$\pm0.2$ (4) & 1.2e+22 & 0.62 (4) & 6.25$\pm5$ (4) & N & Y \\
{\bf CSS081007} & \nodata & 2454596.5 & \nodata & \nodata & 0.146 & 1.1e+21 & 1.77 (5) & 4.45$\pm1.95$ (6) & \nodata & \nodata \\
GQ Mus & 7.2 (7) & 2445352.5 (7) & 18 (7) & 1000 (8) & 0.45 (9) & 3.8e+21  & 0.059375 (10) & 4.8$\pm1$ (9) & N (7) & \nodata \\
IM Nor & 7.84 (11) & 2452289 (2) & 50 (2) & 1150 (12) & 0.8$\pm0.2$ (4) & 8e+21 & 0.102 (13) & 4.25$\pm3.4$ (4) & N & Y \\
{\bf KT Eri} & 5.42 (14) & 2455150.17 (14) & 6.6 (14) & 3000 (15) & 0.08 (15) & 5.5e+20 & \nodata & 6.5 (15) & N & M \\
{\bf LMC 1995} & 10.7 (16) & 2449778.5 (16) & 15$\pm2$ (17) & \nodata & 0.15 (203) & 7.8e+20  & \nodata & 50 & \nodata & \nodata \\
LMC 2000 & 11.45 (18) & 2451737.5 (18) & 9$\pm2$ (19) & 1700 (20) & 0.15 (203) & 7.8e+20  & \nodata & 50 & \nodata & \nodata \\
{\bf LMC 2005} & 11.5 (21) & 2453700.5 (21) & 63 (22) & 900 (23) & 0.15 (203) & 1e+21 & \nodata & 50  & M (24) & \nodata \\
{\bf LMC 2009a} & 10.6 (25) & 2454867.5 (25) & 4$\pm1$  & 3900 (25) & 0.15 (203)  & 5.7e+20 & 1.19 (26) & 50 & N & Y \\
{\bf SMC 2005} & 10.4 (27) & 2453588.5 (27) & \nodata & 3200 (28) & \nodata & 5e+20  & \nodata & 61 & \nodata & \nodata \\
{\bf QY Mus} & 8.1 (29) & 2454739.90 (29) & 60:  & \nodata & 0.71 (30) & 4.2e+21  & \nodata & \nodata & M & \nodata \\
{\bf RS Oph} & 4.5 (31) & 2453779.44 (14) & 7.9 (14) & 3930 (31) & 0.73 (32) & 2.25e+21 & 456 (33) & 1.6$\pm0.3$ (33) & N (34) & Y \\
{\bf U Sco} & 8.05 (35) & 2455224.94 (35) & 1.2 (36) & 7600 (37) & 0.2$\pm0.1$ (4) & 1.2e+21 & 1.23056 (36) & 12$\pm2$ (4) & N & Y \\
{\bf V1047 Cen} & 8.5 (38) & 2453614.5 (39) & 6 (40) & 840 (38) & \nodata & 1.4e+22  & \nodata & \nodata & \nodata & \nodata \\
{\bf V1065 Cen} & 8.2 (41) & 2454123.5 (41) & 11 (42) & 2700 (43) & 0.5$\pm0.1$ (42) & 3.75e+21 & \nodata & 9.05$\pm2.8$ (42) & Y (42) & \nodata \\
V1187 Sco & 7.4 (44) & 2453220.5 (44) & 7: (45) & 3000 (44) & 1.56 (44) & 8.0e+21 & \nodata & 4.9$\pm0.5$ (44) & N & \nodata \\
{\bf V1188 Sco} & 8.7 (46) & 2453577.5 (46) & 7 (40) & 1730 (47) & \nodata & 5.0e+21  & \nodata & 7.5 (39) & \nodata & \nodata \\
{\bf V1213 Cen} & 8.53 (48) & 2454959.5 (48) & 11$\pm2$ (49) & 2300 (50) & 2.07 (30) & 1.0e+22 & \nodata & \nodata & \nodata & \nodata \\
{\bf V1280 Sco} & 3.79 (51) & 2454147.65 (14) & 21 (52) & 640 (53) & 0.36 (54) & 1.6e+21  & \nodata & 1.6$\pm0.4$ (54) & Y (54) & \nodata \\
{\bf V1281 Sco} & 8.8 (55) & 2454152.21 (55) & 15:& 1800 (56) & 0.7 (57) & 3.2e+21 & \nodata & \nodata & N & \nodata \\
{\bf V1309 Sco} & 7.1 (58) & 2454714.5 (58) & 23$\pm2$ (59) & 670 (60) & 1.2 (30) & 4.0e+21 & \nodata & \nodata & \nodata & \nodata \\
{\bf V1494 Aql} & 3.8 (61) & 2451515.5 (61) & 6.6$\pm0.5$ (61) & 1200 (62) & 0.6 (63) & 3.6e+21  & 0.13467 (64) & 1.6$\pm0.1$ (63) & N & \nodata \\
{\bf V1663 Aql} & 10.5 (65) & 2453531.5 (65) & 17 (66) & 1900 (67) & 2: (68) & 1.6e+22  & \nodata & 8.9$\pm3.6$ (69) & N & \nodata \\
V1974 Cyg & 4.3 (70) & 2448654.5 (70) & 17 (71) & 2000 (19) & 0.36$\pm0.04$ (71) & 2.7e+21  & 0.081263 (70) & 1.8$\pm0.1$ (72) & N & \nodata \\
{\bf V2361 Cyg} & 9.3 (73) & 2453412.5 (73) & 6 (40) & 3200 (74) & 1.2: (75) & 7.0e+21 & \nodata & \nodata & Y (40) & \nodata \\
{\bf V2362 Cyg} & 7.8 (76) & 2453831.5 (76) & 9 (77) & 1850 (78) & 0.575$\pm0.015$ (79) & 4.4e+21  & 0.06577 (80) & 7.75$\pm3$ (77) & Y (81) & \nodata \\
{\bf V2467 Cyg} & 6.7 (82) & 2454176.27 (82) & 7 (83) & 950 (82) & 1.5 (84) & 1.4e+22  & 0.159 (85) & 3.1$\pm0.5$ (86) & M (87) & \nodata \\
{\bf V2468 Cyg} & 7.4 (88) & 2454534.2 (88) & 10: & 1000 (88) & 0.77 (89) & 1.0e+22  & 0.242 (90) & \nodata & N & \nodata \\
{\bf V2491 Cyg} & 7.54 (91) & 2454567.86 (91) & 4.6 (92) & 4860 (93) & 0.43 (94) & 4.7e+21  & 0.09580: (95) & 10.5 (96) & N & M \\
V2487 Oph & 9.5 (97) & 2450979.5 (97) & 6.3 (98) & 10000 (98) & 0.38$\pm0.08$ (98) & 2.0e+21 & \nodata & 27.5$\pm3$ (99) & N (100) & Y (101) \\
{\bf V2540 Oph} & 8.5 (102) & 2452295.5 (102) & \nodata & \nodata & \nodata & 2.3e+21 & 0.284781 (103) & 5.2$\pm0.8$ (103) & N & \nodata \\
V2575 Oph & 11.1 (104) & 2453778.8 (104) & 20: & 560 (104) & 1.4 (105) & 3.3e+21 & \nodata & \nodata & N (105) & \nodata \\
{\bf V2576 Oph} & 9.2 (106) & 2453832.5 (106) & 8: & 1470 (106) & 0.25 (107) & 2.6e+21  & \nodata & \nodata & N & \nodata \\
{\bf V2615 Oph} & 8.52 (108) & 2454187.5 (108) & 26.5 (108) & 800 (109) & 0.9 (108) & 3.1e+21  & \nodata & 3.7$\pm0.2$ (108) & Y (110) & \nodata \\
{\bf V2670 Oph} & 9.9 (111) & 2454613.11 (111) & 15: & 600 (112) & 1.3: (113) & 2.9e+21  & \nodata & \nodata & N (114) & \nodata \\
{\bf V2671 Oph} & 11.1 (115) & 2454617.5 (115) & 8: & 1210 (116) & 2.0 (117) & 3.3e+21  & \nodata & \nodata & M (117) & \nodata \\
{\bf V2672 Oph} & 10.0 (118) & 2455060.02 (118) & 2.3 (119) & 8000 (118) & 1.6$\pm0.1$ (119) & 4.0e+21  & \nodata & 19$\pm2$ (119) & \nodata & M \\
V351 Pup & 6.5 (120) & 2448617.5 (120) & 16 (121) & \nodata & 0.72$\pm0.1$ (122) & 6.2e+21 & 0.1182 (123) & 2.7$\pm0.7$ (122) & N & \nodata \\
{\bf V382 Nor} & 8.9 (124) & 2453447.5 (124) & 12 (40) & 1850 (23) & \nodata & 1.7e+22 & \nodata & \nodata & \nodata & \nodata \\
V382 Vel & 2.85 (125) & 2451320.5 (125) & 4.5 (126) & 2400 (126) & 0.05: (126) & 3.4e+21  & 0.146126 (127) & 1.68$\pm0.3$ (126) & N & \nodata \\
{\bf V407 Cyg} & 6.8 (128) & 2455266.314 (128) & 5.9 (129) & 2760 (129) & 0.5$\pm0.05$ (130) & 8.8e+21 & 15595 (131) & 2.7 (131) & \nodata & Y \\
{\bf V458 Vul} & 8.24 (132) & 2454322.39 (132) & 7 (133) & 1750 (134) & 0.6 (135) & 3.6e+21 & 0.06812255 (136) & 8.5$\pm1.8$ (133) & N (135) & \nodata \\
{\bf V459 Vul} & 7.57 (137) & 2454461.5 (137) & 18 (138) & 910 (139) & 1.0 (140) & 5.5e+21  & \nodata & 3.65$\pm1.35$ (138) & Y (140) & \nodata \\
V4633 Sgr & 7.8 (141) & 2450895.5 (141) & 19$\pm3$ (142) & 1700 (143) & 0.21 (142) & 1.4e+21  & 0.125576 (144) & 8.9$\pm2.5$ (142) & N & \nodata \\
{\bf V4643 Sgr} & 8.07 (145) & 2451965.867 (145) & 4.8 (146) & 4700 (147) & 1.67 (148) & 1.4e+22 & \nodata & 3 (148) & N & \nodata \\
{\bf V4743 Sgr} & 5.0 (149) & 2452537.5 (149) & 9 (150) & 2400 (149) & 0.25 (151) & 1.2e+21 & 0.281 (152) & 3.9$\pm0.3$ (151) & N & \nodata \\
{\bf V4745 Sgr} & 7.41 (153) & 2452747.5 (153) & 8.6 (154) & 1600 (155) & 0.1 (154) & 9.0e+20  & 0.20782 (156) & 14$\pm5$ (154) & \nodata & \nodata \\
{\bf V476 Sct} & 10.3 (157) & 2453643.5 (157) & 15 (158) & \nodata & 1.9 (158) & 1.2e+22  & \nodata & 4$\pm1$ (158) & M (159) & \nodata \\
{\bf V477 Sct} & 9.8 (160) & 2453655.5 (160) & 3 (160) & 2900 (161) & 1.2: (162) & 4e+21  & \nodata & \nodata & M (163) & \nodata \\
{\bf V5114 Sgr} & 8.38 (164) & 2453081.5 (164) & 11 (165) & 2000 (23) & \nodata & 1.5e+21  & \nodata & 7.7$\pm0.7$ (165) & N (166) & \nodata \\
{\bf V5115 Sgr} & 7.7 (167) & 2453459.5 (167) & 7 (40) & 1300 (168) & 0.53 (169) & 2.3e+21  & \nodata & \nodata & N (169) & \nodata \\
{\bf V5116 Sgr} & 8.15 (170) & 2453556.91 (170) & 6.5 (171) & 970 (172) & 0.25 (173) & 1.5e+21 & 0.1238 (171) & 11$\pm3$ (173) & N (174) & \nodata \\
{\bf V5558 Sgr} & 6.53 (175) & 2454291.5 (175) & 125 (176) & 1000 (177) & 0.80 (178) & 1.6e+22  & \nodata & 1.3$\pm0.3$ (176) & N (179) & \nodata \\
{\bf V5579 Sgr} & 5.56 (180) & 2454579.62 (180) & 7: & 1500 (23) & 1.2 (181) & 3.3e+21 & \nodata & \nodata & Y (181) & \nodata \\
{\bf V5583 Sgr} & 7.43 (182) & 2455051.07 (182) & 5: & 2300 (182) & 0.39 (30) & 2.0e+21 & \nodata & 10.5 & \nodata & \nodata \\
{\bf V574 Pup} & 6.93 (183) & 2453332.22 (183) & 13 (184) & 2800 (184) & 0.5$\pm0.1$  & 6.2e+21 & \nodata & 6.5$\pm1$  & M (185) & \nodata \\
{\bf V597 Pup} & 7.0 (186) & 2454418.75 (186) & 3: & 1800 (187) & 0.3 (188) & 5.0e+21  & 0.11119 (189) & \nodata & N (188) & \nodata \\
{\bf V598 Pup} & 3.46 (14) & 2454257.79 (14) & 9$\pm1$ (190) & \nodata & 0.16 (190) & 1.4e+21 & \nodata & 2.95$\pm0.8$ (190) & \nodata & \nodata \\
{\bf V679 Car} & 7.55 (191) & 2454797.77 (191) & 20: & \nodata & \nodata & 1.3e+22  & \nodata & \nodata & \nodata & \nodata \\
{\bf V723 Cas} & 7.1 (192) & 2450069.0 (192) & 263 (2) & 600 (193) & 0.5 (194) & 2.35e+21  & 0.69 (195) & 3.86$\pm0.23$ (196) & N & \nodata \\
V838 Her & 5 (197) & 2448340.5 (197) & 2 (198) & \nodata & 0.5$\pm0.1$ (198) & 2.6e+21  & 0.2975 (199) & 3$\pm1$ (198) & Y (200) & \nodata \\
{\bf XMMSL1 J06} & 12 (201) & 2453643.5 (202) & 8$\pm2$ (202) & \nodata & 0.15 (203) & 8.7e+20 & \nodata & 50 & \nodata & \nodata \\
\enddata
\end{deluxetable*}
\end{longrotatetable}

\bibliography{template}{}
\bibliographystyle{aasjournal}

\end{document}

