\documentclass[twocolumn]{aastex631}
\usepackage{cleveref}

% \usepackage{multicol,lipsum,environ}

% \NewEnviron{auxmulticols}[1]{%
%   \ifnum#1<2\relax% Fewer than 2 columns
%     %\vspace{-\baselineskip}% Possible vertical correction
%     \BODY
%   \else% More than 1 column
%     \begin{multicols}{#1}
%       \BODY
%     \end{multicols}%
%   \fi
% }

\newcommand{\vdag}{(v)^\dagger}
\newcommand\aastex{AAS\TeX}
\newcommand\latex{La\TeX}

\newcommand{\mreal}{M_{\text{real}}}
\newcommand{\minst}{M_{\text{inst.}}}

\shorttitle{Path finding with the old breed}
\shortauthors{O. Johnson}

\graphicspath{{./}{figures/}}
\begin{document}
\title{Pathfinding with the Old Breed: Using old clusters of the Milky Way for Galatic Tracing}
\author[0000-0002-5927-0481]{Owen Johnson}
\affiliation{School of Physics, University College Dublin, Ireland}



%% Mark off the abstract in the ``abstract'' environment. 
\begin{abstract}

    Open clusters have been long used as stellar laboratories, and galactic tracers in this study both are used in tandem to investigate the distribution of old ($> 1000$ Myr) open clusters in a milky way. UB photometric data were collected for Berkeley 28, Bochum 2, NGC 2124 and NGC 2155. Each cluster's population was determined using Gaia data and parameterised using MIST and DSEP isochrones. Along with classification based on the Trumpler scheme and cataloguing of the stellar population. These clusters were combined with 266 open clusters from newly available catalogues. It was found that there is an underabundance of old open clusters within the inner galactic disk of the Milky Way despite production outweighing disruptive dynamical forces in the galaxy. It is shown that there is no direct correlation between cluster age and galactic position. Indicating sprawling embedment of older clusters throughout the milky way is due to a layered relationship between internal cluster dynamics and the disk environment. It is deemed that an underabundance of old clusters is due to dispersion due to destructive interactions and misrepresentation due to observational selection. With the old breed of open clusters found to be inflating the galactic disk. 

% This example manuscript is intended to serve as a tutorial and template for
% authors to use when writing their own AAS Journal articles. The manuscript
% includes a history of \aastex\ and documents the new features in the
% previous versions as well as the bug fixes in version 6.31. This
% manuscript includes many figure and table examples to illustrate these new
% features.  Information on features not explicitly mentioned in the article
% can be viewed in the manuscript comments or more extensive online
% documentation. Authors are welcome replace the text, tables, figures, and
% bibliography with their own and submit the resulting manuscript to the AAS
% Journals peer review system.  The first lesson in the tutorial is to remind
% authors that the AAS Journals, the Astrophysical Journal (ApJ), the
% Astrophysical Journal Letters (ApJL), the Astronomical Journal (AJ), and
% the Planetary Science Journal (PSJ) all have a 250 word limit for the 
% abstract\footnote{Abstracts for Research Notes of the American Astronomical 
% Society (RNAAS) are limited to 150 words}.  If you exceed this length the
% Editorial office will ask you to shorten it. This abstract has 182 words.

\end{abstract}


\keywords{Open Clusters --- Galactic Tracing --- Galactic Disk --- MIST Isochrone Fitting}
\section*{}

% \begin{auxmulticols}{1}
%     \lipsum[1-2]
% \end{auxmulticols}

\begin{center}
    \textit{This work is dedicated to my mentor and my friend Noel White, (1951-2021).}
\end{center}

\section{Introduction} \label{sec:intro}

Open clusters have been shown to be an integral part of the astronomers toolbox, readily lending themselves as stellar laboratories. Open clusters are classified as a group of stars around the same age and loosely bound through mutual gravitation. 
Their similar age allows for in depth observation of the stellar evolution.  Through this many attributes of the stellar population can be inferred. As clusters span age ranges from a X to X, many have been present since formation of the disk itself. Through this if clusters of varying ages are examined it's possible to trace out the evolution of the milky way.  \\

Mapping the milky way has always been difficult given the vantage point it can be observed from. This makes it quite difficult to appreciate the shape and dimensions of the milky way. Some of the pioneering studies such as \cite{1785RSPT...75..213H,1918ApJ....48..154S} and \cite{1930LicOB..14..154T} first outline the use of open clusters to map the galaxy. Following with studies like \cite{1970IAUS...38..205B} which pathed the spiral arms of the milky way using open clusters and numerous studies by \cite{1958ZA.....46..176V} which explore the evolution of the galaxies scale height. To more recent studies by X \\

While the precision and accuracy of cluster age estimates are tied to the quality of the observational data and theoretical models the process of estimating cluster age through use of colour-magnitude diagrams is relativity straightforward  and been shown to be tried and true. Even early open cluster catalogues like X and X included distance estimates while more recent catalogues like X and X have provided other parameters such as age, metallicity  and excess colour. Furthermore with the second data release from GAIA (X) presents the most in-depth all sky astrometric and photometric study to date. \\
This increase in available data has allowed for the characterisation of open clusters on mass adding to catalogues such as WEBDA. Determination of all open clusters identified by Gaia is an ongoing task and is being automated using modern techniques and machine learning as shown in studies by X and X. \\

This study used the 1.25 m optical telescope at the Calar Alto Observatory (CAHA) to observe four open clusters from the WEBDA catalogue. The aim of this work was to classify the four observed clusters and infer details of each cluster. Then use this observational cluster classification in tandem with other open clusters from the WEBDA catalogue to trace the paths of clusters in the galactic disk studying both its structure and evolution.

% \section{Targets}

\subsection{Open Cluster Types}

As open clusters span many different distribtuions in both density, size and stellar constituents. Open clusters can contain large stellar agglomerations to just a handfull of stars. While classification systems can vary based on the context of the study the scheme coined by \cite{1930LicOB..14..154T} sees promeinant use. \\ This scheme classifies cluster based on three factors of the stellar population. a) their range of brightness, b) degree of concerntration and c) star population in the cluster. The details of this classification scheme can be seen in \cref{tab:trumpler_scheme}. 

\begin{deluxetable*}{lll}
    \tablecaption{Trumpler classification scheme. \label{tab:trumpler_scheme}}
    \tablehead{
    \colhead{Range of Brightness} & \colhead{Degree of Concerntration} & \colhead{Cluster Population} \\
    \colhead{(a)} & \colhead{(b)} & \colhead{(c)} 
    % \colhead{Number} & \colhead{Number} & \nocolhead{Name} & \colhead{Type} &
    % \multicolumn2c{(kpc)} & \colhead{Constellation} & \colhead{(mag)}
    }
    % \decimalcolnumbers
    % \tablewidth{}
    \startdata
    1 - Majority of stellar objects show similar brightness. & I - Strong central concentration (Detached) & p - Poor $(n < 50)$  \\
    2 - Moderate brightness ranges between stellar objects. & II - Little central concentration (Detached)& m - Medium $(50 < n < 100)$ \\
    3 - Both bright and feint stellar objects  & III - No disenable concentration & r - Rich $(n > 100)$\\ 
      & IV - Clusters not well detached (Strong field concentration) & 
    \enddata
    \tablecomments{Where $n$ denotes the amounts the stellar population in a given cluster. For example Pleiades is a I3rn cluster and Hyades is a II3m cluster. Where the 'n' flag on a classification relates if the cluster shows nebulosity.}
\end{deluxetable*}

\section{Observations} 

\subsection{Target Selection}

\begin{figure*}
  \gridline{\fig{figures/target_selection.pdf}{0.8\textwidth}{}
            }
  \caption{Aitoff projection of targets in terms of galactic co-ordinates, longitude ($l$) and laititude ($b$). Targets observed at CAHA are observed in X, the orginal proposal targets shown in X and studies by \cite{2019A&A...623A.108B} and 
  \label{fig:TargetSelection}}
\end{figure*}
  

\subsection{Photometry}

\section{Supplementary Data}

This study compiles X clusters to aid the observational data. When searching for studies to compliment this work use Gaia's second data release (DR2) was set given preference. The reason for the use of supplementary data was to provide a more varying survey of the galactic disk. The first data set implented was 269 clusters analysed and catalogued by \cite{2019A&A...623A.108B}. This dataset contains large sample of clusters analysed from Gaia DR2, with each of the clusters containg a high degree homogenity amoung the stelllar population. The cluster populations were determined using Bayesian methods of statistics along with DR2 astrometric data. In doing this the probablity of each star being a member of each clusters was approximately X. The parameters of each cluster was found using PARSEC isochrones \citep{2012MNRAS.427..127B}. This data set worked well to fill out a sample size in the galactic disk as seen in \cref{fig:TargetSelection}. Although this survey contained a good amount of clusters with varying age it lacked some the ancient clusters of the milkey way. To supplement this gap the Swift UVOT near-infrared isochrone study using Gaia DR2 \citep{2019yCat..51580035S} was added to the pool of clusters for analysis.

\begin{figure}[ht!]
  \plotone{supplementary_data_plot.pdf}
  \caption{Distance against the log age of both observed targets, proposed targets and supplementary targets.   \label{fig:supplementarydatafig}}
\end{figure}


\begin{deluxetable*}{lccccc}
  \tablecaption{Results of Trumpler classification on observed targets. \label{tab:trumpler_results}}
  \tablehead{
  \colhead{Target} & \colhead{$\Delta V_{mag}$} & \colhead{$\Delta B_{mag}$} & \colhead{$\sigma_s$} & \colhead{Population $n$} 
  }
  \decimalcolnumbers
  % \tablewidth{}
  \startdata
  Berkeley 28  &  &  &  &  & m \\ 
  Bochum 2 &  &  &  &  & \\ 
  NGC2324 & 12 & 12 & 12 & 333 & r \\
  NGC2324 &  &  &  &  & r \\ 
  \enddata
  \tablecomments{Where $n$ denotes the amounts the stellar population in a given cluster. For example Pleiades is a I3rn cluster and Hyades is a II3m cluster. Where the 'n' flag on a classification relates if the cluster shows nebulosity.}
\end{deluxetable*}



\begin{acknowledgments}
    \section*{Acknowledgments}
    I would like to give my thanks to Dr Antonio Martin-Carillo for his patience, warmth and depth throughout this project. He made this project an absolute joy from start to finish. I want to give a special thanks to Camin Mac Cionna for his marvellous code, coffee and 'career' advice, to my open cluster partner in crime, Eoin Fitzpatrick, for making my exploration of the universe a little less lonely and to Sean J. Brennan for the wise insights and advice in cycling and astrophysics a-like. \\ 
    Furthermore, I would like to thank my peers for making the last four years of learning about the cosmos the best experience I've ever had. I wish them every luck and success in their future careers. \\
To A-a-ron, Ben, Ciara, Cian, Hugh, Joe, Laura, Sharon, and Tiernan, thank you for taking care of me for the past few years. Your friendship is both invaluable and necessary. 
 \\ Finally, to my family, for their unwavering support and help in the pursuit of my dreams always. Thank you for making all of this possible. \\

\end{acknowledgments}

\clearpage 

\bibliography{sample631}{}
\bibliographystyle{aasjournal}



%% This command is needed to show the entire author+affiliation list when
%% the collaboration and author truncation commands are used.  It has to
%% go at the end of the manuscript.
%\allauthors

%% Include this line if you are using the \added, \replaced, \deleted
%% commands to see a summary list of all changes at the end of the article.
%\listofchanges



\clearpage

\appendix

\section{Stellar Catalogs} \label{sec:star_catalog}
\startlongtable
\begin{deluxetable*}{lCCCCC}
    \tablecaption{Stellar catalog for NGC 2324.}
    \tablewidth{0pt}
    \tablehead{
    \colhead{\textit{no.}} & \colhead{RA J2000} &  \colhead{DEC J2000} &  \colhead{B Magnitude} &  \colhead{V Magnitude} &  \colhead{BV Magnitude}
    } 
    \startdata
    \multicolumn{6}{c}{\textit{Population Size: $n = 251$}} \\ \hline 
    1 & 106.0320 & 1.0263 & 16.979 \pm 0.2136 & 16.222 \pm 0.1053 & -0.308 \pm  0.2213 \\ 
    2 & 105.9987 & 1.0177 & 17.434 \pm 0.2181 & 16.054 \pm 0.1044 & 0.314  \pm 0.2252 \\ 
    3 & 106.0556 & 1.0286 & 17.084 \pm 0.2145 & 16.007 \pm 0.1042 & 0.011  \pm 0.2216 \\ 
    4 & 106.0382 & 1.0512 & 17.094 \pm 0.2146 & 15.944 \pm 0.1039 & 0.084  \pm 0.2215 \\ 
    5 & 105.9826 & 1.0289 & 17.342 \pm 0.2171 & 15.849 \pm 0.1035 & 0.428  \pm 0.2238 \\ 
    6 & 106.0143 & 1.0555 & 17.057 \pm 0.2142 & 15.835 \pm 0.1034 & 0.156  \pm 0.2210 \\ 
    7 & 106.0161 & 1.0613 & 17.110 \pm 0.2147 & 15.825 \pm 0.1034 & 0.219  \pm 0.2214 \\ 
    8 & 106.0291 & 1.0476 & 16.778 \pm 0.2122 & 15.814 \pm 0.1033 & -0.102 \pm  0.2190 \\ 
    9 & 106.0309 & 1.0594 & 16.918 \pm 0.2131 & 15.761 \pm 0.1031 & 0.090  \pm 0.2198 \\ 
    10 & 106.0442 & 1.0013 & 16.991 \pm 0.2137 & 15.721 \pm 0.1029 & 0.204   \pm  0.2202 \\ 
    11 & 106.0495 & 1.0284 & 16.669 \pm 0.2116 & 15.717 \pm 0.1029 & -0.114  \pm  0.2182 \\ 
    12 & 106.0369 & 1.0464 & 16.785 \pm 0.2122 & 15.707 \pm 0.1029 & 0.012   \pm 0.2188 \\ 
    13 & 105.9968 & 1.0264 & 16.638 \pm 0.2114 & 15.703 \pm 0.1029 & -0.131  \pm  0.2180 \\ 
    14 & 105.9931 & 1.0016 & 17.073 \pm 0.2144 & 15.692 \pm 0.1028 & 0.315   \pm 0.2209 \\ 
    15 & 106.0407 & 1.0536 & 16.925 \pm 0.2132 & 15.667 \pm 0.1027 & 0.192   \pm 0.2197 \\ 
    16 & 106.0274 & 1.0323 & 16.767 \pm 0.2121 & 15.667 \pm 0.1027 & 0.034   \pm 0.2186 \\ 
    17 & 106.0336 & 0.9892 & 16.787 \pm 0.2122 & 15.572 \pm 0.1024 & 0.148   \pm 0.2186 \\ 
    18 & 106.0542 & 1.0754 & 17.463 \pm 0.2185 & 15.559 \pm 0.1023 & 0.838   \pm 0.2246 \\ 
    19 & 106.0313 & 1.0692 & 16.783 \pm 0.2122 & 15.554 \pm 0.1023 & 0.162   \pm 0.2185 \\ 
    20 & 105.9841 & 1.0214 & 17.096 \pm 0.2146 & 15.509 \pm 0.1022 & 0.522   \pm 0.2207 \\ 
    21 & 106.0528 & 1.0410 & 16.584 \pm 0.2111 & 15.473 \pm 0.1020 & 0.045   \pm 0.2173 \\ 
    22 & 106.0193 & 1.0270 & 16.304 \pm 0.2101 & 15.447 \pm 0.1020 & -0.209  \pm  0.2163 \\ 
    23 & 106.0343 & 1.0663 & 16.540 \pm 0.2109 & 15.423 \pm 0.1019 & 0.050   \pm 0.2171 \\ 
    24 & 105.9778 & 0.9913 & 17.114 \pm 0.2147 & 15.423 \pm 0.1019 & 0.625   \pm 0.2208 \\ 
    25 & 106.0358 & 1.0662 & 16.608 \pm 0.2112 & 15.416 \pm 0.1019 & 0.126   \pm 0.2174 \\ 
    26 & 106.0525 & 1.0444 & 16.579 \pm 0.2111 & 15.407 \pm 0.1018 & 0.107   \pm 0.2172 \\ 
    27 & 106.0053 & 0.9881 & 16.662 \pm 0.2115 & 15.399 \pm 0.1018 & 0.197   \pm 0.2176 \\ 
    28 & 106.0157 & 1.0514 & 16.487 \pm 0.2107 & 15.396 \pm 0.1018 & 0.025   \pm 0.2168 \\ 
    29 & 106.0043 & 0.9921 & 16.697 \pm 0.2117 & 15.395 \pm 0.1018 & 0.236   \pm 0.2178 \\ 
    30 & 105.9810 & 0.9939 & 16.951 \pm 0.2134 & 15.384 \pm 0.1018 & 0.501   \pm 0.2194 \\ 
    31 & 105.9710 & 1.0659 & 17.103 \pm 0.2146 & 15.378 \pm 0.1017 & 0.660   \pm 0.2206 \\ 
    32 & 105.9634 & 1.0453 & 17.115 \pm 0.2147 & 15.367 \pm 0.1017 & 0.683   \pm 0.2207 \\ 
    33 & 105.9842 & 1.0098 & 17.045 \pm 0.2141 & 15.365 \pm 0.1017 & 0.614   \pm 0.2201 \\ 
    34 & 105.9665 & 1.0210 & 17.170 \pm 0.2153 & 15.343 \pm 0.1016 & 0.761   \pm 0.2212 \\ 
    35 & 106.0567 & 1.0868 & 16.980 \pm 0.2136 & 15.320 \pm 0.1016 & 0.594   \pm 0.2195 \\ 
    36 & 106.0499 & 1.0336 & 16.353 \pm 0.2102 & 15.309 \pm 0.1015 & -0.022  \pm  0.2162 \\ 
    37 & 106.0553 & 1.0949 & 17.153 \pm 0.2151 & 15.299 \pm 0.1015 & 0.788   \pm 0.2209 \\ 
    38 & 106.0631 & 1.0076 & 17.054 \pm 0.2142 & 15.293 \pm 0.1015 & 0.695   \pm 0.2201 \\ 
    39 & 106.0533 & 1.0059 & 16.826 \pm 0.2125 & 15.292 \pm 0.1015 & 0.468   \pm 0.2184 \\ 
    40 & 106.0621 & 1.0766 & 16.961 \pm 0.2134 & 15.287 \pm 0.1015 & 0.609   \pm 0.2193 \\ 
    41 & 105.9629 & 1.0175 & 17.165 \pm 0.2152 & 15.275 \pm 0.1014 & 0.824   \pm 0.2210 \\ 
    42 & 105.9882 & 1.0846 & 16.948 \pm 0.2133 & 15.271 \pm 0.1014 & 0.611   \pm 0.2192 \\ 
    43 & 105.9800 & 1.0797 & 17.306 \pm 0.2166 & 15.265 \pm 0.1014 & 0.975   \pm 0.2224 \\ 
    44 & 106.0726 & 1.0145 & 17.267 \pm 0.2162 & 15.255 \pm 0.1014 & 0.946   \pm 0.2220 \\ 
    45 & 106.0252 & 1.0508 & 16.359 \pm 0.2103 & 15.224 \pm 0.1013 & 0.069   \pm 0.2161 \\ 
    46 & 106.0578 & 1.0877 & 17.004 \pm 0.2138 & 15.208 \pm 0.1012 & 0.730   \pm 0.2196 \\ 
    47 & 106.0040 & 1.0989 & 17.150 \pm 0.2151 & 15.200 \pm 0.1012 & 0.884   \pm 0.2208 \\ 
    48 & 106.0241 & 1.0882 & 16.611 \pm 0.2113 & 15.199 \pm 0.1012 & 0.346   \pm 0.2171 \\ 
    49 & 106.0488 & 1.0838 & 17.187 \pm 0.2154 & 15.192 \pm 0.1012 & 0.930   \pm 0.2211 \\ 
    50 & 106.0142 & 1.0749 & 16.290 \pm 0.2101 & 15.182 \pm 0.1011 & 0.041   \pm 0.2159 \\ 
    51 & 106.0510 & 1.0519 & 16.221 \pm 0.2099 & 15.155 \pm 0.1011 & 0.000   \pm 0.2157 \\ 
    52 & 105.9881 & 1.0819 & 16.670 \pm 0.2116 & 15.153 \pm 0.1011 & 0.450   \pm 0.2173 \\ 
    53 & 105.9833 & 1.0301 & 16.319 \pm 0.2101 & 15.137 \pm 0.1010 & 0.116   \pm 0.2159 \\ 
    54 & 106.0790 & 1.0563 & 16.877 \pm 0.2128 & 15.136 \pm 0.1010 & 0.675   \pm 0.2185 \\ 
    55 & 105.9600 & 1.0502 & 17.051 \pm 0.2142 & 15.133 \pm 0.1010 & 0.852   \pm 0.2198 \\ 
    56 & 106.0393 & 0.9967 & 16.211 \pm 0.2099 & 15.130 \pm 0.1010 & 0.015   \pm 0.2156 \\ 
    57 & 106.0184 & 1.0557 & 16.071 \pm 0.2096 & 15.091 \pm 0.1009 & -0.087  \pm  0.2153 \\ 
    58 & 105.9890 & 1.0160 & 16.734 \pm 0.2119 & 15.082 \pm 0.1009 & 0.585   \pm 0.2176 \\ 
    59 & 106.0212 & 1.0846 & 16.200 \pm 0.2098 & 15.080 \pm 0.1009 & 0.054   \pm 0.2155 \\ 
    60 & 106.0489 & 1.0496 & 16.100 \pm 0.2097 & 15.072 \pm 0.1009 & -0.038  \pm  0.2154 \\ 
    61 & 106.0272 & 0.9848 & 16.282 \pm 0.2100 & 15.067 \pm 0.1008 & 0.149   \pm 0.2157 \\ 
    62 & 106.0359 & 1.0554 & 17.004 \pm 0.2138 & 15.064 \pm 0.1008 & 0.874   \pm 0.2194 \\ 
    63 & 106.0579 & 0.9862 & 16.380 \pm 0.2103 & 15.054 \pm 0.1008 & 0.260   \pm 0.2160 \\ 
    64 & 106.0080 & 1.0195 & 16.052 \pm 0.2096 & 15.043 \pm 0.1008 & -0.057  \pm  0.2153 \\ 
    65 & 106.0373 & 1.0413 & 15.956 \pm 0.2095 & 15.004 \pm 0.1007 & -0.114  \pm  0.2151 \\ 
    66 & 106.0495 & 1.0200 & 15.988 \pm 0.2095 & 14.992 \pm 0.1006 & -0.070  \pm  0.2151 \\ 
    67 & 106.0253 & 1.0730 & 16.233 \pm 0.2099 & 14.991 \pm 0.1006 & 0.177   \pm 0.2155 \\ 
    68 & 106.0947 & 1.0782 & 17.212 \pm 0.2157 & 14.969 \pm 0.1006 & 1.177   \pm 0.2211 \\ 
    69 & 106.0885 & 1.0511 & 16.815 \pm 0.2124 & 14.959 \pm 0.1006 & 0.790   \pm 0.2179 \\ 
    70 & 105.9918 & 1.0284 & 16.011 \pm 0.2096 & 14.949 \pm 0.1005 & -0.004  \pm  0.2151 \\ 
    71 & 105.9795 & 1.0037 & 16.627 \pm 0.2113 & 14.946 \pm 0.1005 & 0.615   \pm 0.2168 \\ 
    72 & 106.0498 & 1.1099 & 16.864 \pm 0.2127 & 14.941 \pm 0.1005 & 0.857   \pm 0.2182 \\ 
    73 & 106.0286 & 1.0142 & 15.965 \pm 0.2095 & 14.933 \pm 0.1005 & -0.034  \pm  0.2151 \\ 
    74 & 106.0892 & 1.0796 & 16.981 \pm 0.2136 & 14.931 \pm 0.1005 & 0.983   \pm 0.2190 \\ 
    75 & 106.0203 & 1.0151 & 15.862 \pm 0.2095 & 14.914 \pm 0.1005 & -0.118  \pm  0.2150 \\ 
    76 & 106.0427 & 1.0309 & 15.802 \pm 0.2095 & 14.912 \pm 0.1005 & -0.176  \pm  0.2150 \\ 
    77 & 106.0696 & 0.9896 & 16.461 \pm 0.2106 & 14.908 \pm 0.1004 & 0.487   \pm 0.2161 \\ 
    78 & 106.0572 & 1.0266 & 15.953 \pm 0.2095 & 14.904 \pm 0.1004 & -0.017  \pm  0.2150 \\ 
    79 & 106.0735 & 1.0033 & 16.656 \pm 0.2115 & 14.899 \pm 0.1004 & 0.691   \pm 0.2169 \\ 
    80 & 106.0926 & 0.9878 & 16.976 \pm 0.2136 & 14.882 \pm 0.1004 & 1.028   \pm 0.2189 \\ 
    81 & 106.0505 & 1.1124 & 16.674 \pm 0.2116 & 14.872 \pm 0.1004 & 0.736   \pm 0.2170 \\ 
    82 & 105.9830 & 1.0221 & 16.010 \pm 0.2096 & 14.862 \pm 0.1003 & 0.081   \pm 0.2150 \\ 
    83 & 105.9863 & 1.0748 & 16.194 \pm 0.2098 & 14.859 \pm 0.1003 & 0.269   \pm 0.2153 \\ 
    84 & 106.0783 & 1.0460 & 16.616 \pm 0.2113 & 14.858 \pm 0.1003 & 0.691   \pm 0.2167 \\ 
    85 & 106.0198 & 1.0015 & 15.927 \pm 0.2095 & 14.844 \pm 0.1003 & 0.017   \pm 0.2149 \\ 
    86 & 105.9758 & 1.0841 & 16.436 \pm 0.2105 & 14.843 \pm 0.1003 & 0.527   \pm 0.2159 \\ 
    87 & 106.0525 & 1.0353 & 15.834 \pm 0.2095 & 14.827 \pm 0.1003 & -0.059  \pm  0.2149 \\ 
    88 & 105.9800 & 1.0532 & 16.493 \pm 0.2107 & 14.825 \pm 0.1003 & 0.602   \pm 0.2161 \\ 
    89 & 106.0122 & 1.0208 & 15.883 \pm 0.2095 & 14.824 \pm 0.1002 & -0.007  \pm  0.2149 \\ 
    90 & 105.9896 & 1.0169 & 16.358 \pm 0.2103 & 14.823 \pm 0.1002 & 0.470   \pm 0.2157 \\ 
    91 & 106.0998 & 1.0303 & 17.299 \pm 0.2166 & 14.823 \pm 0.1002 & 1.410   \pm 0.2218 \\ 
    92 & 106.0106 & 1.0310 & 15.776 \pm 0.2095 & 14.805 \pm 0.1002 & -0.095  \pm  0.2149 \\ 
    93 & 106.0762 & 1.0942 & 16.805 \pm 0.2123 & 14.799 \pm 0.1002 & 0.939   \pm 0.2177 \\ 
    94 & 105.9695 & 1.0514 & 16.129 \pm 0.2097 & 14.795 \pm 0.1002 & 0.269   \pm 0.2151 \\ 
    95 & 106.0243 & 1.0295 & 15.678 \pm 0.2095 & 14.791 \pm 0.1002 & -0.179  \pm  0.2149 \\ 
    96 & 106.0046 & 1.0664 & 15.863 \pm 0.2095 & 14.780 \pm 0.1002 & 0.017   \pm 0.2149 \\ 
    97 & 105.9905 & 1.0382 & 15.845 \pm 0.2095 & 14.770 \pm 0.1001 & 0.009   \pm 0.2148 \\ 
    98 & 106.0104 & 1.0013 & 15.859 \pm 0.2095 & 14.767 \pm 0.1001 & 0.027   \pm 0.2148 \\ 
    99 & 105.9869 & 0.9967 & 16.002 \pm 0.2095 & 14.766 \pm 0.1001 & 0.170   \pm 0.2149 \\ 
    100 & 106.0589 & 1.1020 & 16.292 \pm 0.2101 & 14.757 \pm  0.1001 & 0.469 \pm 0.2541 \\ 
    101 & 105.9769 & 1.1117 & 16.511 \pm 0.2108 & 14.753 \pm  0.1001 & 0.692 \pm 0.2161 \\ 
    102 & 106.0193 & 1.0993 & 16.125 \pm 0.2097 & 14.746 \pm  0.1001 & 0.313 \pm 0.2150 \\ 
    103 & 106.0757 & 0.9951 & 16.416 \pm 0.2104 & 14.741 \pm  0.1001 & 0.609 \pm 0.2158 \\ 
    104 & 106.0194 & 1.0640 & 15.749 \pm 0.2095 & 14.721 \pm  0.1000 & -0.038 \pm 0.2148 \\ 
    105 & 106.0060 & 1.0872 & 16.234 \pm 0.2099 & 14.673 \pm  0.0999 & 0.494 \pm 0.2152 \\ 
    106 & 106.0338 & 1.0576 & 15.639 \pm 0.2096 & 14.668 \pm  0.0999 & -0.095 \pm 0.2148 \\ 
    107 & 105.9565 & 1.0858 & 16.366 \pm 0.2103 & 14.664 \pm  0.0999 & 0.636 \pm 0.2155 \\ 
    108 & 105.9657 & 1.1165 & 16.697 \pm 0.2117 & 14.644 \pm  0.0999 & 0.987 \pm 0.2169 \\ 
    109 & 106.0116 & 1.0761 & 15.666 \pm 0.2096 & 14.642 \pm  0.0999 & -0.042 \pm 0.2148 \\ 
    110 & 106.0953 & 1.0309 & 16.246 \pm 0.2099 & 14.635 \pm  0.0998 & 0.545 \pm 0.2152 \\ 
    111 & 106.0915 & 1.0710 & 16.430 \pm 0.2105 & 14.634 \pm  0.0998 & 0.730 \pm 0.2157 \\ 
    112 & 106.0778 & 1.0906 & 16.204 \pm 0.2099 & 14.629 \pm  0.0998 & 0.509 \pm 0.2151 \\ 
    113 & 106.0436 & 1.0672 & 15.670 \pm 0.2096 & 14.624 \pm  0.0998 & -0.020 \pm 0.2148 \\ 
    114 & 106.0180 & 1.0221 & 15.525 \pm 0.2097 & 14.619 \pm  0.0998 & -0.160 \pm 0.2149 \\ 
    115 & 106.0387 & 1.0906 & 16.355 \pm 0.2102 & 14.612 \pm  0.0998 & 0.677 \pm 0.2154 \\ 
    116 & 105.9582 & 1.0737 & 16.554 \pm 0.2110 & 14.596 \pm  0.0998 & 0.892 \pm 0.2162 \\ 
    117 & 105.9915 & 1.0753 & 15.752 \pm 0.2095 & 14.593 \pm  0.0998 & 0.094 \pm 0.2147 \\ 
    118 & 106.0355 & 1.0545 & 15.793 \pm 0.2095 & 14.582 \pm  0.0997 & 0.145 \pm 0.2147 \\ 
    119 & 106.0694 & 1.0329 & 15.904 \pm 0.2095 & 14.581 \pm  0.0997 & 0.257 \pm 0.2147 \\ 
    120 & 106.0980 & 1.0607 & 16.513 \pm 0.2108 & 14.570 \pm  0.0997 & 0.877 \pm 0.2160 \\ 
    121 & 106.0975 & 1.1044 & 16.797 \pm 0.2123 & 14.561 \pm  0.0997 & 1.171 \pm 0.2174 \\ 
    122 & 106.0845 & 1.0639 & 15.955 \pm 0.2095 & 14.551 \pm  0.0997 & 0.338 \pm 0.2147 \\ 
    123 & 106.0555 & 1.0483 & 15.617 \pm 0.2096 & 14.543 \pm  0.0997 & 0.008 \pm 0.2148 \\ 
    124 & 106.0585 & 1.0015 & 15.800 \pm 0.2095 & 14.537 \pm  0.0997 & 0.197 \pm 0.2146 \\ 
    125 & 106.0358 & 1.0788 & 15.611 \pm 0.2096 & 14.530 \pm  0.0996 & 0.015 \pm 0.2148 \\ 
    126 & 105.9800 & 1.0601 & 15.710 \pm 0.2095 & 14.529 \pm  0.0996 & 0.115 \pm 0.2147 \\ 
    127 & 106.0781 & 1.1021 & 16.475 \pm 0.2107 & 14.526 \pm  0.0996 & 0.883 \pm 0.2158 \\ 
    128 & 106.0187 & 1.0158 & 15.570 \pm 0.2097 & 14.518 \pm  0.0996 & -0.014 \pm  0.2148 \\ 
    129 & 106.0003 & 0.9951 & 15.633 \pm 0.2096 & 14.501 \pm  0.0996 & 0.067  \pm 0.2147 \\ 
    130 & 106.0409 & 1.0481 & 15.436 \pm 0.2099 & 14.500 \pm  0.0996 & -0.130 \pm  0.2150 \\ 
    131 & 106.0980 & 1.0964 & 16.628 \pm 0.2113 & 14.474 \pm  0.0995 & 1.088  \pm 0.2164 \\ 
    132 & 106.0719 & 0.9990 & 15.835 \pm 0.2095 & 14.469 \pm  0.0995 & 0.299  \pm 0.2146 \\ 
    133 & 106.0175 & 1.0704 & 15.448 \pm 0.2099 & 14.463 \pm  0.0995 & -0.081 \pm 0.2149 \\ 
    134 & 106.0884 & 1.0050 & 15.960 \pm 0.2095 & 14.458 \pm  0.0995 & 0.436 \pm 0.2146 \\ 
    135 & 105.9741 & 1.0465 & 15.628 \pm 0.2096 & 14.442 \pm  0.0995 & 0.120 \pm 0.2147 \\ 
    136 & 106.0433 & 1.1170 & 15.850 \pm 0.2095 & 14.421 \pm  0.0995 & 0.363 \pm 0.2145 \\ 
    137 & 106.0517 & 1.1186 & 15.792 \pm 0.2095 & 14.383 \pm  0.0994 & 0.343 \pm 0.2145 \\ 
    138 & 106.0503 & 1.1029 & 15.714 \pm 0.2095 & 14.383 \pm  0.0994 & 0.266 \pm 0.2145 \\ 
    139 & 106.0129 & 1.0572 & 15.346 \pm 0.2101 & 14.368 \pm  0.0994 & -0.088 \pm 0.2151 \\ 
    140 & 106.0645 & 1.0586 & 15.611 \pm 0.2096 & 14.354 \pm  0.0993 & 0.191 \pm 0.2146 \\ 
    141 & 105.9636 & 1.1125 & 16.119 \pm 0.2097 & 14.338 \pm  0.0993 & 0.715 \pm 0.2147 \\ 
    142 & 106.0645 & 1.0316 & 15.457 \pm 0.2098 & 14.335 \pm  0.0993 & 0.057 \pm 0.2148 \\ 
    143 & 106.0337 & 1.0542 & 15.912 \pm 0.2095 & 14.286 \pm  0.0992 & 0.560 \pm 0.2144 \\ 
    144 & 106.0659 & 1.0556 & 15.685 \pm 0.2095 & 14.282 \pm  0.0992 & 0.337 \pm 0.2145 \\ 
    145 & 106.0470 & 1.0313 & 15.225 \pm 0.2104 & 14.278 \pm  0.0992 & -0.118 \pm 0.2153 \\ 
    146 & 105.9660 & 1.0862 & 15.769 \pm 0.2095 & 14.277 \pm  0.0992 & 0.426 \pm 0.2144 \\ 
    147 & 106.0228 & 1.0597 & 16.246 \pm 0.2099 & 14.267 \pm  0.0992 & 0.913 \pm 0.2149 \\ 
    148 & 105.9637 & 1.0927 & 15.725 \pm 0.2095 & 14.262 \pm  0.0992 & 0.397 \pm 0.2144 \\ 
    149 & 106.0091 & 1.1110 & 16.244 \pm 0.2099 & 14.253 \pm  0.0992 & 0.925 \pm 0.2149 \\ 
    150 & 106.0741 & 1.0975 & 15.516 \pm 0.2097 & 14.229 \pm  0.0991 & 0.221 \pm 0.2146 \\ 
    151 & 106.0576 & 1.0252 & 15.209 \pm 0.2105 & 14.215 \pm  0.0991 & -0.073 \pm  0.2153 \\ 
    152 & 106.0334 & 1.0417 & 14.215 \pm 0.2152 & 14.203 \pm  0.0991 & -1.054 \pm  0.2199 \\ 
    153 & 106.0514 & 0.9968 & 15.367 \pm 0.2100 & 14.197 \pm  0.0991 & 0.104  \pm 0.2149 \\ 
    154 & 106.0524 & 1.1203 & 16.124 \pm 0.2097 & 14.181 \pm  0.0991 & 0.877  \pm 0.2146 \\ 
    155 & 106.0752 & 1.0700 & 16.000 \pm 0.2095 & 14.177 \pm  0.0991 & 0.757  \pm 0.2144 \\ 
    156 & 106.0431 & 1.0650 & 15.232 \pm 0.2104 & 14.173 \pm  0.0991 & -0.007 \pm  0.2152 \\ 
    157 & 106.0409 & 0.9858 & 15.260 \pm 0.2103 & 14.159 \pm  0.0990 & 0.034  \pm 0.2151 \\ 
    158 & 106.0198 & 1.0527 & 15.772 \pm 0.2095 & 14.152 \pm  0.0990 & 0.554  \pm 0.2143 \\ 
    159 & 106.0283 & 1.0747 & 15.206 \pm 0.2105 & 14.145 \pm  0.0990 & -0.005 \pm  0.2153 \\ 
    160 & 106.0535 & 1.0180 & 15.221 \pm 0.2104 & 14.139 \pm  0.0990 & 0.016  \pm 0.2152 \\ 
    161 & 106.0385 & 1.0793 & 15.161 \pm 0.2106 & 14.085 \pm  0.0989 & 0.010  \pm 0.2154 \\ 
    162 & 106.0024 & 1.0842 & 15.254 \pm 0.2103 & 14.078 \pm  0.0989 & 0.110  \pm 0.2151 \\ 
    163 & 106.0465 & 1.0458 & 15.084 \pm 0.2109 & 14.071 \pm  0.0989 & -0.053 \pm  0.2156 \\ 
    164 & 105.9602 & 1.1131 & 15.595 \pm 0.2096 & 14.051 \pm  0.0989 & 0.477  \pm 0.2144 \\ 
    165 & 106.0401 & 1.0596 & 15.077 \pm 0.2109 & 14.048 \pm  0.0989 & -0.036 \pm  0.2156 \\ 
    166 & 106.0096 & 0.9904 & 15.180 \pm 0.2105 & 14.042 \pm  0.0989 & 0.072  \pm 0.2153 \\ 
    167 & 106.0342 & 1.0259 & 14.978 \pm 0.2112 & 14.040 \pm  0.0989 & -0.128 \pm  0.2160 \\ 
    168 & 105.9843 & 1.0058 & 15.159 \pm 0.2106 & 14.017 \pm  0.0988 & 0.076  \pm 0.2153 \\ 
    169 & 106.0588 & 1.0190 & 15.894 \pm 0.2095 & 14.014 \pm  0.0988 & 0.814  \pm 0.2142 \\ 
    170 & 106.0842 & 1.0595 & 15.297 \pm 0.2102 & 14.001 \pm  0.0988 & 0.230  \pm 0.2149 \\ 
    171 & 106.0761 & 1.0389 & 15.148 \pm 0.2106 & 13.993 \pm  0.0988 & 0.089  \pm 0.2154 \\ 
    172 & 105.9799 & 1.0022 & 15.113 \pm 0.2108 & 13.989 \pm  0.0988 & 0.058  \pm 0.2155 \\ 
    173 & 106.0738 & 1.0686 & 15.215 \pm 0.2104 & 13.969 \pm  0.0988 & 0.181  \pm 0.2151 \\ 
    174 & 106.0007 & 1.0140 & 14.970 \pm 0.2113 & 13.963 \pm  0.0988 & -0.059 \pm  0.2160 \\ 
    175 & 106.0907 & 1.0583 & 15.422 \pm 0.2099 & 13.904 \pm  0.0987 & 0.452  \pm 0.2146 \\ 
    176 & 106.0041 & 1.0138 & 15.230 \pm 0.2104 & 13.902 \pm  0.0987 & 0.261  \pm 0.2151 \\ 
    177 & 105.9887 & 1.0022 & 14.982 \pm 0.2112 & 13.898 \pm  0.0987 & 0.019 \pm 0.2159 \\ 
    178 & 106.0584 & 1.0184 & 15.389 \pm 0.2100 & 13.897 \pm  0.0987 & 0.426 \pm 0.2147 \\ 
    179 & 106.0181 & 1.0145 & 14.893 \pm 0.2116 & 13.886 \pm  0.0987 & -0.058\pm  0.2162 \\ 
    180 & 105.9647 & 1.1025 & 15.268 \pm 0.2103 & 13.879 \pm  0.0987 & 0.323 \pm 0.2149 \\ 
    181 & 106.0678 & 1.0223 & 15.613 \pm 0.2096 & 13.868 \pm  0.0986 & 0.679 \pm 0.2143 \\ 
    182 & 105.9682 & 1.0181 & 14.993 \pm 0.2112 & 13.857 \pm  0.0986 & 0.069 \pm 0.2158 \\ 
    183 & 106.0331 & 1.0331 & 14.787 \pm 0.2120 & 13.822 \pm  0.0986 & -0.101\pm  0.2166 \\ 
    184 & 105.9976 & 1.0363 & 14.834 \pm 0.2118 & 13.814 \pm  0.0986 & -0.046 \pm  0.2164 \\ 
    185 & 106.0240 & 1.1174 & 15.046 \pm 0.2110 & 13.794 \pm  0.0985 & 0.186  \pm 0.2156 \\ 
    186 & 106.0583 & 1.0340 & 14.768 \pm 0.2121 & 13.769 \pm  0.0985 & -0.067 \pm  0.2167 \\ 
    187 & 106.0455 & 1.0493 & 14.770 \pm 0.2121 & 13.759 \pm  0.0985 & -0.055 \pm  0.2167 \\ 
    188 & 106.0000 & 1.0750 & 14.826 \pm 0.2119 & 13.745 \pm  0.0985 & 0.015  \pm 0.2164 \\ 
    189 & 105.9948 & 1.0968 & 14.911 \pm 0.2115 & 13.723 \pm  0.0985 & 0.122  \pm 0.2161 \\ 
    190 & 106.0439 & 1.0398 & 14.733 \pm 0.2123 & 13.706 \pm  0.0984 & -0.039 \pm  0.2168 \\ 
    191 & 106.0677 & 1.0189 & 14.936 \pm 0.2114 & 13.705 \pm  0.0984 & 0.165  \pm 0.2159 \\ 
    192 & 105.9917 & 1.0253 & 14.765 \pm 0.2121 & 13.689 \pm  0.0984 & 0.010  \pm 0.2167 \\ 
    193 & 106.0773 & 1.0171 & 14.846 \pm 0.2118 & 13.671 \pm  0.0984 & 0.109  \pm 0.2163 \\ 
    194 & 106.0129 & 1.0222 & 14.678 \pm 0.2126 & 13.657 \pm  0.0984 & -0.046 \pm  0.2171 \\ 
    195 & 106.0043 & 1.0295 & 14.678 \pm 0.2126 & 13.651 \pm  0.0984 & -0.039 \pm  0.2171 \\ 
    196 & 106.0304 & 1.0291 & 14.696 \pm 0.2125 & 13.646 \pm  0.0984 & -0.016 \pm  0.2170 \\ 
    197 & 106.0350 & 1.0280 & 14.538 \pm 0.2133 & 13.638 \pm  0.0984 & -0.166 \pm  0.2177 \\ 
    198 & 106.0388 & 1.1202 & 15.064 \pm 0.2109 & 13.628 \pm  0.0984 & 0.370  \pm 0.2154 \\ 
    199 & 106.0456 & 0.9960 & 14.748 \pm 0.2122 & 13.612 \pm  0.0983 & 0.069  \pm 0.2167 \\ 
    200 & 106.0485 & 1.0860 & 16.056 \pm 0.2096 & 13.580 \pm  0.0983 & 1.409  \pm 0.2141 \\ 
    201 & 106.0764 & 0.9984 & 14.729 \pm 0.2123 & 13.556 \pm  0.0983 & 0.106  \pm 0.2168 \\ 
    202 & 106.0223 & 1.0604 & 14.671 \pm 0.2126 & 13.548 \pm  0.0983 & 0.057  \pm 0.2170 \\ 
    203 & 105.9938 & 1.0694 & 14.602 \pm 0.2129 & 13.532 \pm  0.0983 & 0.004  \pm 0.2174 \\ 
    204 & 106.0321 & 1.0136 & 14.535 \pm 0.2133 & 13.492 \pm  0.0982 & -0.023 \pm 0.2177 \\ 
    205 & 106.0443 & 1.0383 & 14.500 \pm 0.2135 & 13.482 \pm  0.0982 & -0.048 \pm 0.2179 \\ 
    206 & 106.0882 & 1.0696 & 14.796 \pm 0.2120 & 13.454 \pm  0.0982 & 0.276  \pm 0.2164 \\ 
    207 & 106.0641 & 1.0832 & 14.568 \pm 0.2131 & 13.439 \pm  0.0982 & 0.063  \pm 0.2175 \\ 
    208 & 106.0291 & 1.0333 & 14.415 \pm 0.2140 & 13.428 \pm  0.0982 & -0.079 \pm 0.2183 \\ 
    209 & 106.0229 & 1.0266 & 14.448 \pm 0.2138 & 13.417 \pm  0.0982 & -0.035 \pm 0.2181 \\ 
    210 & 106.0020 & 1.0358 & 14.424 \pm 0.2139 & 13.358 \pm  0.0981 & 0.000  \pm 0.2182 \\ 
    211 & 106.0593 & 1.0050 & 14.422 \pm 0.2139 & 13.260 \pm  0.0980 & 0.096  \pm 0.2182 \\ 
    212 & 106.0706 & 1.1025 & 14.610 \pm 0.2129 & 13.235 \pm  0.0980 & 0.310  \pm 0.2172 \\ 
    213 & 106.0496 & 1.0402 & 14.225 \pm 0.2151 & 13.233 \pm  0.0980 & -0.074 \pm 0.2194 \\ 
    214 & 106.0169 & 1.0001 & 14.281 \pm 0.2147 & 13.228 \pm  0.0980 & -0.013 \pm 0.2190 \\ 
    215 & 105.9710 & 1.0889 & 14.416 \pm 0.2139 & 13.180 \pm  0.0980 & 0.170 \pm 0.2182 \\ 
    216 & 105.9817 & 1.1130 & 14.408 \pm 0.2140 & 13.164 \pm  0.0979 & 0.178 \pm 0.2183 \\ 
    217 & 106.0095 & 1.0843 & 14.253 \pm 0.2149 & 13.129 \pm  0.0979 & 0.057 \pm 0.2192 \\ 
    218 & 106.0470 & 1.0873 & 14.245 \pm 0.2150 & 13.043 \pm  0.0979 & 0.135 \pm 0.2192 \\ 
    219 & 106.0530 & 1.0459 & 14.185 \pm 0.2153 & 13.033 \pm  0.0978 & 0.086 \pm 0.2195 \\ 
    220 & 105.9752 & 1.1037 & 14.521 \pm 0.2134 & 13.022 \pm  0.0978 & 0.434 \pm 0.2176 \\ 
    221 & 106.0302 & 1.0893 & 14.981 \pm 0.2112 & 12.948 \pm  0.0978 & 0.967 \pm 0.2155 \\ 
    222 & 106.0501 & 1.0747 & 14.205 \pm 0.2152 & 12.932 \pm  0.0978 & 0.207 \pm 0.2194 \\ 
    223 & 106.0682 & 1.0232 & 14.178 \pm 0.2154 & 12.923 \pm  0.0978 & 0.188 \pm 0.2196 \\ 
    224 & 106.0939 & 1.1049 & 14.700 \pm 0.2124 & 12.880 \pm  0.0977 & 0.754 \pm 0.2167 \\ 
    225 & 106.0450 & 1.0694 & 13.967 \pm 0.2168 & 12.855 \pm  0.0977 & 0.045 \pm 0.2209 \\ 
    226 & 106.0719 & 1.0437 & 13.962 \pm 0.2168 & 12.826 \pm  0.0977 & 0.069 \pm 0.2209 \\ 
    227 & 106.0484 & 1.0869 & 14.093 \pm 0.2159 & 12.804 \pm  0.0977 & 0.224 \pm 0.2201 \\ 
    228 & 106.0237 & 1.0703 & 13.903 \pm 0.2172 & 12.782 \pm  0.0977 & 0.054 \pm 0.2213 \\ 
    229 & 106.0564 & 1.0346 & 14.443 \pm 0.2138 & 12.780 \pm  0.0977 & 0.598 \pm 0.2179 \\ 
    230 & 106.0243 & 1.0098 & 14.401 \pm 0.2140 & 12.779 \pm  0.0977 & 0.556 \pm 0.2182 \\ 
    231 & 106.0213 & 1.0483 & 13.877 \pm 0.2174 & 12.756 \pm  0.0977 & 0.055 \pm 0.2215 \\ 
    232 & 106.0221 & 1.0498 & 14.312 \pm 0.2146 & 12.620 \pm  0.0976 & 0.626 \pm 0.2187 \\ 
    233 & 106.0979 & 0.9930 & 13.864 \pm 0.2175 & 12.610 \pm  0.0976 & 0.188 \pm 0.2216 \\ 
    234 & 106.0155 & 1.0043 & 14.250 \pm 0.2149 & 12.585 \pm  0.0976 & 0.599 \pm 0.2190 \\ 
    235 & 106.0294 & 1.0663 & 13.773 \pm 0.2182 & 12.579 \pm  0.0976 & 0.128 \pm 0.2222 \\ 
    236 & 106.0385 & 1.0540 & 13.705 \pm 0.2187 & 12.529 \pm  0.0975 & 0.110 \pm 0.2227 \\ 
    237 & 105.9645 & 1.0739 & 13.698 \pm 0.2187 & 12.514 \pm  0.0975 & 0.118 \pm 0.2227 \\ 
    238 & 106.0665 & 1.0144 & 13.736 \pm 0.2185 & 12.505 \pm  0.0975 & 0.165 \pm 0.2225 \\ 
    239 & 106.0648 & 1.0807 & 13.682 \pm 0.2189 & 12.495 \pm  0.0975 & 0.121 \pm 0.2229 \\ 
    240 & 106.0614 & 1.0185 & 14.168 \pm 0.2155 & 12.446 \pm  0.0975 & 0.656 \pm 0.2195 \\ 
    241 & 106.0046 & 1.1011 & 14.186 \pm 0.2153 & 12.445 \pm  0.0975 & 0.675 \pm 0.2194 \\ 
    242 & 106.0342 & 1.0884 & 14.151 \pm 0.2156 & 12.413 \pm  0.0975 & 0.673 \pm 0.2196 \\ 
    243 & 106.0104 & 1.0195 & 13.443 \pm 0.2207 & 12.381 \pm  0.0975 & -0.004 \pm 0.2247 \\ 
    244 & 105.9915 & 1.0714 & 14.050 \pm 0.2162 & 12.311 \pm  0.0974 & 0.674 \pm 0.2202 \\ 
    245 & 105.9882 & 1.0872 & 13.186 \pm 0.2229 & 12.028 \pm  0.0973 & 0.092 \pm 0.2267 \\ 
    246 & 105.9898 & 1.0247 & 13.183 \pm 0.2229 & 11.917 \pm  0.0973 & 0.201 \pm 0.2267 \\ 
    247 & 105.9651 & 1.1109 & 13.741 \pm 0.2184 & 11.812 \pm  0.0973 & 0.863 \pm 0.2223 \\ 
    248 & 106.0157 & 1.0393 & 13.167 \pm 0.2230 & 11.787 \pm  0.0973 & 0.314 \pm 0.2268 \\ 
    249 & 105.9571 & 1.0948 & 13.300 \pm 0.2219 & 11.776 \pm  0.0973 & 0.458 \pm 0.2257 \\ 
    250 & 106.0672 & 1.0204 & 13.436 \pm 0.2208 & 11.528 \pm  0.0972 & 0.842 \pm 0.2246 \\ 
    251 & 106.0140 & 1.0670 & 10.760 \pm 0.2476 & 9.315  \pm 0.0971 & 0.380 \pm 0.2510 \\ 
    \enddata
    \tablecomments{ }
\end{deluxetable*}

\end{document}

% End of file `sample631.tex'.
